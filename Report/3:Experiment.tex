\section{Experiment}
\paragraph{}
This experiment followed strict patterns to ensure that it would not affect the servers of each of the tested solutions. No real data was used in the tests. Each solution got an account created and fake data inserted into it.


\subsection{Testing environment}
\paragraph{}
The research will be conducted within a client-server environment.

\subsubsection{Servers}
\paragraph{}
In this research we will be using five servers with the following specifications:\\~\\
\textbf{Physical server}
\begin{itemize}
	\item Intel Xeon processor
	\item 8GB ram
	\item SSD + HDD for storage
\end{itemize}

 For the four physical servers we will be using each server to setup a separate functionality. Additional tools will be used, as well multiple logging and bench marking tools that allow us to keep a detailed log on the resource consumption and performance measurement. The fifth physical server is used for virtualization.

\textbf{virtual server}
\paragraph{}
For the virtual server we will be using Xen 4.8.2 as hypervisor to implements the VMs. On top of the hypervisor there is a Dom0.

\paragraph{}
To test the solutions the following tools were used to setup a testing environment:
\begin{itemize}
    \item Ubuntu 16.04.2
    \begin{itemize}
        \item Virtual Box - 5.0.40
        \begin{itemize}
            \item Windows 7 and 10 x64 - Fully updated
            \begin{itemize}
                \item Process Explorer - 16.21
                \item Windows 7/10 SDK - 7.1/10.0.15063.400
                \item Proxifier - 3.31
            \end{itemize}
        \end{itemize}
        \item Burp Suite - 1.7.23
        \item Apktool - 2.2.2
        \item Dex2jar - Commit dd9d722
        \item Jd\_Gui - Commit acd511f
    \end{itemize}
    \item Android - 7.1.2
    \begin{itemize}
        \item Proxydroid - 2.7.7
    \end{itemize}
\end{itemize}

\subsection{Tool}

%%tools we use


\clearpage
\subsection{Methodology}
\paragraph{}
In this project we followed a methodology that starts from basic tests to increasingly complex ones on both virtual and physical servers. The steps we followed were:
\begin{itemize}
    \item \textbf{Basic setup tests:}
 In the basic setup tests we used only the Apache HTTP server benchmarking tool to experiment the web server alone by itself without the complete setup as illustrated in the \textbf{\textit{figure 1}}. 
 We targeted only the static web page of our osCommerce store.
 \begin{figure}[H]
    \centering
    \includegraphics[width=6cm]{Pictures/simple.PNG}
    \caption{Basic Setup Test Methodology}
    \label{fig:QQ3}
\end{figure}
   
    
    
    \item \textbf{Full setup tests:}
 For the complete setup test we used both the Apache HTTP server benchmarking tool and Paessler webserver Stress tool to experiment the complete setup environment which includes proxy, web, and database servers as shown in \textbf{\textit{figure 2}}. 
 We targeted the proxy server which acts as load balancer between our two webservers. And these two webservers interact directly with the database server. Which leads to a complete setup tests.   
 \begin{figure}[H]
    \centering
    \includegraphics[width=12cm]{Pictures/complex.PNG}
    \caption{Full Setup Test Methodology}
    \label{fig:QQ3}
\end{figure}


    \item ..
\end{itemize}
\subsubsection{Paessler Webserver Stress Tool}
\paragraph{}
We first decided to do \ref{fig:QQ3}.

NOTE: Below should be an image with the test 
%%                           > WEB 1 \
%%    Stress tool to Proxy  <         > Database 
%%                           > WEB 2 /

\begin{figure}[H]
    \centering
    \includegraphics[width=6cm]{Figures/Screenshot.png}
    \caption{Screenshot}
    \label{fig:QQ3}
\end{figure}
\paragraph{}
In a password manager it is not different. For this test we imputed strings of varied lengths and with has many symbols as possible into all of the fields. Lastly, we tried to crash the program by imputing a very long string pattern of 20 thousand characters.

\subsubsection{Apache Benchmark}

NOTE: Below should be an image with the test 
%%                  > WEB 1 \
%%    AB to Proxy  <         > Database 
%%                  > WEB 2 /

NOTE: Below should be an image with the test 
%%      AB to      > WEB 1 